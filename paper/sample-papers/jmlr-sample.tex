 % use the "wcp" class option for workshop and conference
 % proceedings
 %\documentclass[gray]{jmlr} % test grayscale version
 \documentclass[tablecaption=bottom]{jmlr}% journal article
 %\documentclass[tablecaption=bottom,wcp]{jmlr} % W&CP article

 % The following packages will be automatically loaded:
 % amsmath, amssymb, natbib, graphicx, url, algorithm2e

 %\usepackage{rotating}% for sideways figures and tables
 %\usepackage{longtable}% for long tables

 % The booktabs package is used by this sample document
 % (it provides \toprule, \midrule and \bottomrule).
 % Remove the next line if you don't require it.
\usepackage{booktabs}
 % The siunitx package is used by this sample document
 % to align numbers in a column by their decimal point.
 % Remove the next line if you don't require it.
\usepackage[load-configurations=version-1]{siunitx} % newer version
 %\usepackage{siunitx}
\usepackage{multirow}

\usepackage{tikz,pgfplots}%,pgfplots,color,amsmath,amssymb}
%\usepackage{caption}
%\usepackage{subcaption}
\usepackage[capitalise]{cleveref}

 % The following command is just for this sample document:
\newcommand{\cs}[1]{\texttt{\char`\\#1}}% remove this in your real article

 % Define an unnumbered theorem just for this sample document for
 % illustrative purposes:
\theorembodyfont{\upshape}
\theoremheaderfont{\scshape}
\theorempostheader{:}
\theoremsep{\newline}
\newtheorem*{note}{Note}

 % change the arguments, as appropriate, in the following:
\jmlrvolume{1}
\jmlryear{2010}
\jmlrsubmitted{submission date}
\jmlrpublished{publication date}
\jmlrworkshop{workshop title} % W&CP title

 % The optional argument of \title is used in the header
\title[Impact of Nonconformity Functions on Efficiency of Conformal Classifiers]{
Impact of Model-Agnostic Nonconformity Functions on Efficiency of Conformal Classifiers: an Extensive Study
%Full Title of Article\titlebreak This Title Has A Line Break
}

 % Anything in the title that should appear in the main title but 
 % not in the article's header or the volume's table of
 % contents should be placed inside \titletag{}

 %\title{Title of the Article\titletag{\thanks{Some footnote}}}


 % Use \Name{Author Name} to specify the name.
 % If the surname contains spaces, enclose the surname
 % in braces, e.g. \Name{John {Smith Jones}} similarly
 % if the name has a "von" part, e.g \Name{Jane {de Winter}}.
 % If the first letter in the forenames is a diacritic
 % enclose the diacritic in braces, e.g. \Name{{\'E}louise Smith}

 % \thanks must come after \Name{...} not inside the argument for
 % example \Name{John Smith}\nametag{\thanks{A note}} NOT \Name{John
 % Smith\thanks{A note}}

 % Anything in the name that should appear in the title but not in the 
 % article's header or footer or in the volume's
 % table of contents should be placed inside \nametag{}

 % Two authors with the same address
%  \author{
%  \Name{Author Name1\nametag{\thanks{A note}}} \Email{abc@sample.com}\and
%  \Name{Author Name2} \Email{xyz@sample.com}\\
%  \addr Address}


 % Three or more authors with the same address:
 % \author{\Name{Author Name1} \Email{an1@sample.com}\\
 %  \Name{Author Name2} \Email{an2@sample.com}\\
 %  \Name{Author Name3} \Email{an3@sample.com}\\
 %  \Name{Author Name4} \Email{an4@sample.com}\\
 %  \Name{Author Name5} \Email{an5@sample.com}\\
 %  \Name{Author Name6} \Email{an6@sample.com}\\
 %  \Name{Author Name7} \Email{an7@sample.com}\\
 %  \Name{Author Name8} \Email{an8@sample.com}\\
 %  \Name{Author Name9} \Email{an9@sample.com}\\
 %  \Name{Author Name10} \Email{an10@sample.com}\\
 %  \Name{Author Name11} \Email{an11@sample.com}\\
 %  \Name{Author Name12} \Email{an12@sample.com}\\
 %  \Name{Author Name13} \Email{an13@sample.com}\\
 %  \Name{Author Name14} \Email{an14@sample.com}\\
 %  \addr Address}


 % Authors with different addresses:
  \author{\Name{Marharyta Aleksandrova} \Email{marharyta.aleksandrova@\{uni.lu,gmail.com\}}\\
  \addr University of Luxembourg, 2 avenue de l'Université, L-4365 Esch-sur-Alzette, Luxembourg
  \AND
  \Name{Oleg Chertov} \Email{chertov@i.ua}\\
  \addr National Technical University of Ukraine "Igor Sikorsky Kyiv Polytechnic Institute", \\
  Department of Applied Mathematics, 14-A Politekhnichna St, 03056 Kyiv, Ukraine
 }

\editor{Editor's name}
 %\editors{Editor One and Editor Two}% for multiple editors

\begin{document}

\maketitle

\begin{abstract}
The property of Conformal prediction to guarantee the required accuracy rate makes this framework attractive in various practical applications. However, this property is achieved at a price of reduction in precision. In the case of conformal classification, for example, the systems can output multiple class labels instead of one. It is also known from the literature that the choice of nonconformity function has a major impact on the efficiency of conformal classifiers. Recently it was shown that different model-agnostic nonconformity functions result in conformal classifiers with different characteristics. In the case of Neural Network being used as a baseline classifier, the non-conformity function based on hinge loss allows minimizing the average number of predicted labels, while the non-conformity function based on margin results in a larger fraction of singleton predictions. In this work, we aim to further extend this study. We perform our evaluation using 8 different classification algorithms and discuss when the observed relationship holds or not. Additionally, we analyze the characteristics of hinge loss and margin and give suggestions on how to combine the properties of these two non-conformity functions.
\end{abstract}
\begin{keywords}
Conformal classification,
Nonconformity functions,
Efficiency
\end{keywords}

%\input{sections/old-text}

\section{Experimental results}
\label{sec:experiments}

\subsection{Balance dataset}


\begin{table}[htbp]
\scriptsize
 % The first argument is the label.
 % The caption goes in the second argument, and the table contents
 % go in the third argument.
\floatconts
  {tab:balance-results}%
  {\caption{balance dataset}}%
  {
\begin{tabular}{l|lll|lll|lll|lll|lll}
             & \multicolumn{3}{c|}{$\epsilon=0.01$} & \multicolumn{3}{c|}{$\epsilon=0.05$} & \multicolumn{3}{c|}{$\epsilon=0.1$} & \multicolumn{3}{c|}{$\epsilon=0.15$} & \multicolumn{3}{c}{$\epsilon=0.2$} \\
\hline
\hline
ip           &            & $-$        & $-$*       &            & $-$        & $-$*       &            & $-$        & $-$        &            & $+$        & $-$        &            & $-$        & $-$         \\
ip\_m        & $+$        &            & $-$*       & $+$        &            & $-$*       & $+$        &            & $-$        & $-$        &            & $-$        & $+$        &            & $-$         \\
m            & $+$*       & $+$*       &            & $+$*       & $+$*       &            & $+$        & $+$        &            & $+$        & $+$        &            & $+$        & $+$        &             \\
\hline
\textbf{SVM} & ip         & ip\_m      & m          & ip         & ip\_m      & m          & ip         & ip\_m      & m          & ip         & ip\_m      & m          & ip         & ip\_m      & m           \\
\hline
ip           &            & $-$        & $+$        &            & $-$        & $-$        &            & $-$        & $+$        &            & $-$        & $+$        &            & $+$        & $+$         \\
ip\_m        & $+$        &            & $+$        & $+$        &            & $+$        & $+$        &            & $+$        & $+$        &            & $+$        & $-$        &            & $+$         \\
m            & $-$        & $-$        &            & $+$        & $-$        &            & $-$        & $-$        &            & $-$        & $-$        &            & $-$        & $-$        &             \\
\hline
\hline
ip           &            &            & $-$        &            & $-$        & $-$*       &            & $-$        & $-$*       &            & $-$        & $-$*       &            & $-$        & $-$*        \\
ip\_m        &            &            & $-$        & $+$        &            & $-$*       & $+$        &            & $-$*       & $+$        &            & $-$*       & $+$        &            & $-$*        \\
m            & $+$        & $+$        &            & $+$*       & $+$*       &            & $+$*       & $+$*       &            & $+$*       & $+$*       &            & $+$*       & $+$*       &             \\
\hline
\textbf{DT}  & ip         & ip\_m      & m          & ip         & ip\_m      & m          & ip         & ip\_m      & m          & ip         & ip\_m      & m          & ip         & ip\_m      & m           \\
\hline
ip           &            &            & $-$        &            & $-$        & $-$*       &            & $-$        & $-$*       &            & $-$        & $-$        &            & $-$        & $-$         \\
ip\_m        &            &            & $-$        & $+$        &            & $-$*       & $+$        &            & $-$*       & $+$        &            & $-$        & $+$        &            & $-$         \\
m            & $+$        & $+$        &            & $+$*       & $+$*       &            & $+$*       & $+$*       &            & $+$        & $+$        &            & $+$        & $+$        &             \\
\hline
\hline
ip           &            &            & $-$*       &            & $-$*       & $-$*       &            & $-$        & $-$*       &            &            & $-$*       &            & $-$        & $-$*        \\
ip\_m        &            &            & $-$*       & $+$*       &            & $-$*       & $+$        &            & $-$*       &            &            & $-$*       & $+$        &            & $-$*        \\
m            & $+$*       & $+$*       &            & $+$*       & $+$*       &            & $+$*       & $+$*       &            & $+$*       & $+$*       &            & $+$*       & $+$*       &             \\
\hline
\textbf{KNN} & ip         & ip\_m      & m          & ip         & ip\_m      & m          & ip         & ip\_m      & m          & ip         & ip\_m      & m          & ip         & ip\_m      & m           \\
\hline
ip           &            &            & $-$*       &            & $-$*       & $-$*       &            & $-$        & $-$*       &            &            & $-$*       &            & $-$        & $-$*        \\
ip\_m        &            &            & $-$*       & $+$*       &            & $-$*       & $+$        &            & $-$*       &            &            & $-$*       & $+$        &            & $-$*        \\
m            & $+$*       & $+$*       &            & $+$*       & $+$*       &            & $+$*       & $+$*       &            & $+$*       & $+$*       &            & $+$*       & $+$*       &             \\
\hline
\hline
ip           &            &            & $-$*       &            & $-$*       & $-$*       &            & $-$*       & $-$*       &            & $-$*       & $-$*       &            & $-$*       & $-$*        \\
ip\_m        &            &            & $-$*       & $+$*       &            & $-$*       & $+$*       &            & $-$*       & $+$*       &            & $-$*       & $+$*       &            & $-$         \\
m            & $+$*       & $+$*       &            & $+$*       & $+$*       &            & $+$*       & $+$*       &            & $+$*       & $+$*       &            & $+$*       & $+$        &             \\
\hline
\textbf{Ada} & ip         & ip\_m      & m          & ip         & ip\_m      & m          & ip         & ip\_m      & m          & ip         & ip\_m      & m          & ip         & ip\_m      & m           \\
\hline
ip           &            &            & $-$*       &            & $-$*       & $-$*       &            & $-$*       & $-$*       &            & $-$*       & $-$*       &            & $-$*       & $-$*        \\
ip\_m        &            &            & $-$*       & $+$*       &            & $-$        & $+$*       &            & $+$        & $+$*       &            & $+$        & $+$*       &            & $+$         \\
m            & $+$*       & $+$*       &            & $+$*       & $+$        &            & $+$*       & $-$        &            & $+$*       & $-$        &            & $+$*       & $-$        &             \\
\hline
\hline
ip           &            &            & $-$*       &            & $-$*       & $-$*       &            & $-$        & $-$        &            &            & $-$        &            &            & $-$         \\
ip\_m        &            &            & $-$*       & $+$*       &            & $-$*       & $+$        &            & $-$        &            &            & $-$        &            &            & $-$         \\
m            & $+$*       & $+$*       &            & $+$*       & $+$*       &            & $+$        & $+$        &            & $+$        & $+$        &            & $+$        & $+$        &             \\
\hline
\textbf{GNB} & ip         & ip\_m      & m          & ip         & ip\_m      & m          & ip         & ip\_m      & m          & ip         & ip\_m      & m          & ip         & ip\_m      & m           \\
\hline
ip           &            &            & $-$*       &            & $-$*       & $-$*       &            & $-$        & $-$        &            &            & $+$        &            &            & $+$         \\
ip\_m        &            &            & $-$*       & $+$*       &            & $-$*       & $+$        &            & $-$        &            &            & $+$        &            &            & $+$         \\
m            & $+$*       & $+$*       &            & $+$*       & $+$*       &            & $+$        & $+$        &            & $-$        & $-$        &            & $-$        & $-$        &             \\
\hline
\hline
ip           &            & $-$        & $-$*       &            & $-$        & $-$        &            & $-$        & $-$        &            & $+$        & $-$        &            & $-$        & $+$         \\
ip\_m        & $+$        &            & $-$*       & $+$        &            & $-$        & $+$        &            & $-$        & $-$        &            & $-$        & $+$        &            & $+$         \\
m            & $+$*       & $+$*       &            & $+$        & $+$        &            & $+$        & $+$        &            & $+$        & $+$        &            & $-$        & $-$        &             \\
\hline
\textbf{MPR} & ip         & ip\_m      & m          & ip         & ip\_m      & m          & ip         & ip\_m      & m          & ip         & ip\_m      & m          & ip         & ip\_m      & m           \\
\hline
ip           &            & $-$        & $-$*       &            & $-$        & $-$        &            & $-$        & $-$        &            & $-$        & $+$        &            & $+$        & $-$         \\
ip\_m        & $+$        &            & $-$*       & $+$        &            & $-$        & $+$        &            & $-$        & $+$        &            & $+$        & $-$        &            & $-$         \\
m            & $+$*       & $+$*       &            & $+$        & $+$        &            & $+$        & $+$        &            & $-$        & $-$        &            & $+$        & $+$        &             \\
\hline
\hline
ip           &            & $+$        & $-$*       &            & $-$*       & $-$*       &            & $-$        & $-$        &            & $-$        & $-$        &            & $+$        & $-$         \\
ip\_m        & $-$        &            & $-$*       & $+$*       &            & $-$*       & $+$        &            & $+$        & $+$        &            & $-$        & $-$        &            & $-$         \\
m            & $+$*       & $+$*       &            & $+$*       & $+$*       &            & $+$        & $-$        &            & $+$        & $+$        &            & $+$        & $+$        &             \\
\hline
\textbf{RF}  & ip         & ip\_m      & m          & ip         & ip\_m      & m          & ip         & ip\_m      & m          & ip         & ip\_m      & m          & ip         & ip\_m      & m           \\
\hline
ip           &            & $+$        & $-$*       &            & $-$*       & $-$*       &            & $-$        & $-$        &            & $-$        & $-$        &            & $-$        & $-$         \\
ip\_m        & $-$        &            & $-$*       & $+$*       &            & $-$*       & $+$        &            & $+$        & $+$        &            & $+$        & $+$        &            & $-$         \\
m            & $+$*       & $+$*       &            & $+$*       & $+$*       &            & $+$        & $-$        &            & $+$        & $-$        &            & $+$        & $+$        &             \\
\hline
\hline
ip           &            &            & $-$*       &            & $-$        & $-$        &            & $-$        & $+$        &            &            & $+$        &            &            & $-$         \\
ip\_m        &            &            & $-$*       & $+$        &            & $+$        & $+$        &            & $+$        &            &            & $+$        &            &            & $-$         \\
m            & $+$*       & $+$*       &            & $+$        & $-$        &            & $-$        & $-$        &            & $-$        & $-$        &            & $+$        & $+$        &             \\
\hline
\textbf{QDA} & ip         & ip\_m      & m          & ip         & ip\_m      & m          & ip         & ip\_m      & m          & ip         & ip\_m      & m          & ip         & ip\_m      & m           \\
\hline
ip           &            &            & $-$*       &            & $-$        & $+$*       &            & $-$        & $+$        &            &            & $-$        &            &            & $+$         \\
ip\_m        &            &            & $-$*       & $+$        &            & $+$*       & $+$        &            & $+$        &            &            & $-$        &            &            & $+$         \\
m            & $+$*       & $+$*       &            & $-$*       & $-$*       &            & $-$        & $-$        &            & $+$        & $+$        &            & $-$        & $-$        &             \\
\hline
\hline
\end{tabular}

  }
\end{table}



\begin{table}[htbp]
\scriptsize
 % The first argument is the label.
 % The caption goes in the second argument, and the table contents
 % go in the third argument.
\floatconts
  {tab:balance-results}%
  {\caption{Example table}}%
  {
\begin{tabular}{l|lll|lll|lll|lll|lll}
             & \multicolumn{3}{c|}{$\epsilon=0.01$} & \multicolumn{3}{c|}{$\epsilon=0.05$} & \multicolumn{3}{c|}{$\epsilon=0.1$} & \multicolumn{3}{c}{$\epsilon=0.15$} & \multicolumn{3}{c|}{$\epsilon=0.2$} \\
\hline
\textbf{SVM} & ip         & ip\_m      & m          & ip         & ip\_m      & m          & ip         & ip\_m      & m          & ip         & ip\_m      & m          & ip         & ip\_m      & m           \\

\hline
\textbf{DT}  & ip         & ip\_m      & m          & ip         & ip\_m      & m          & ip         & ip\_m      & m          & ip         & ip\_m      & m          & ip         & ip\_m      & m           \\

\hline
\textbf{KNN} & ip         & ip\_m      & m          & ip         & ip\_m      & m          & ip         & ip\_m      & m          & ip         & ip\_m      & m          & ip         & ip\_m      & m           \\

\hline
\textbf{Ada} & ip         & ip\_m      & m          & ip         & ip\_m      & m          & ip         & ip\_m      & m          & ip         & ip\_m      & m          & ip         & ip\_m      & m           \\

\hline
\textbf{GNB} & ip         & ip\_m      & m          & ip         & ip\_m      & m          & ip         & ip\_m      & m          & ip         & ip\_m      & m          & ip         & ip\_m      & m           \\

\hline
\textbf{MPR} & ip         & ip\_m      & m          & ip         & ip\_m      & m          & ip         & ip\_m      & m          & ip         & ip\_m      & m          & ip         & ip\_m      & m           \\

\hline
\textbf{RF}  & ip         & ip\_m      & m          & ip         & ip\_m      & m          & ip         & ip\_m      & m          & ip         & ip\_m      & m          & ip         & ip\_m      & m           \\

\hline
\textbf{QDA} & ip         & ip\_m      & m          & ip         & ip\_m      & m          & ip         & ip\_m      & m          & ip         & ip\_m      & m          & ip         & ip\_m      & m           \\

\hline
\end{tabular}

  }
\end{table}

\acks{Acknowledgements go here.}

\bibliography{jmlr-sample}

\appendix

\section{First Appendix}\label{apd:first}

This is the first appendix.

\section{Second Appendix}\label{apd:second}

This is the second appendix.

\end{document}
