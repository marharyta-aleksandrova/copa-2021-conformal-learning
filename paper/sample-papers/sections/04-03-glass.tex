%\subsection{glass}

%Results for \textit{glass} dataset are presented in \cref{fig:glass} and \cref{tab:glass}. The
%dataset has 6 classes and it is unbalanced \cref{fig:glass:dist-class}. The performance of
%algorithms is now worse with baseline error ranging from 25\% for RF to 93\% for QDA, see
%\cref{fig:glass:baseline-error}. The difference between the effectiveness of different conformal
%classifiers is also clearly visible on the plots.
%
%We can notice that \textit{margin} is clearly preferred in cases when DT, KNN and QDA are used as baseline classifiers, see \cref{fig:glass:oneC-SVM-DT}, \cref{fig:glass:avgC-SVM-DT}, \cref{fig:glass:oneC-KNN-Ada}, \cref{fig:glass:avgC-KNN-Ada}, \cref{fig:glass:oneC-RF-QDA},
%and \cref{fig:glass:avgC-RF-QDA} and the results in \cref{tab:glass}. For the rest of the
%classifiers the previous observed pattern holds: \textit{margin} is better for $oneC$ and
%\textit{inverse probability} is better for $avgC$. In all these cases, \textit{ip\_m} allows to improve $oneC$ obtained by the \textit{inverse probability} function and improves both \textit{inverse probability} and \textit{margin} in terms of $avgC$. In case of MPR classifier, we can observe improvement in terms of both metrics when \textit{ip\_m} is used. It means that this metric is the best for conformal predictor based on MPR, see \cref{fig:glass:oneC-GNB-MPR}
%and \cref{fig:glass:avgC-GNB-MPR}.

\begin{figure}[htbp]
\floatconts
  {fig:glass}
  {\caption{Results for \textit{glass}: \textit{M} - dashed line,
  \textit{IP} - dash and dot line, \textit{IP\_M} - thin solid line}}
  {%
    \subfigure[Distribution between classes]{
    \label{fig:glass:dist-class}%
    \includeteximage{plots/glass-dist-class}
    }%
    \qquad
    \subfigure[Baseline error, $b\_err$]{\label{fig:glass:baseline-error}%
      \includeteximage{plots/glass-baseline-error}
      }
    \qquad
    \subfigure[$oneC$: SVM and DT]{\label{fig:glass:oneC-SVM-DT}%
      \includeteximage{plots/glass-oneC-0}
      %\includeteximage{plots/glass-dist-class}
      }
    \qquad
    \subfigure[$avgC$: SVM and DT]{\label{fig:glass:avgC-SVM-DT}%
      \includeteximage{plots/glass-avgC-0}
      }
    \qquad
    \subfigure[$oneC$: KNN and Ada]{\label{fig:glass:oneC-KNN-Ada}%
      \includeteximage{plots/glass-oneC-2}
      }
    \qquad
    \subfigure[$avgC$: KNN and Ada]{\label{fig:glass:avgC-KNN-Ada}%
      \includeteximage{plots/glass-avgC-2}
      }
    \qquad
    \subfigure[$oneC$: GNB and MPR]{\label{fig:glass:oneC-GNB-MPR}%
      \includeteximage{plots/glass-oneC-4}
      }
    \qquad
    \subfigure[$avgC$: GNB and MPR]{\label{fig:glass:avgC-GNB-MPR}%
      \includeteximage{plots/glass-avgC-4}
      }
    \qquad
    \subfigure[$oneC$: RF and QDA]{\label{fig:glass:oneC-RF-QDA}%
      \includeteximage{plots/glass-oneC-6}
      }
    \qquad
    \subfigure[$avgC$: RF and QDA]{\label{fig:glass:avgC-RF-QDA}%
      \includeteximage{plots/glass-avgC-6}
      }
  }
\end{figure}



\begin{table}[htbp]
\scriptsize
 % The first argument is the label.
 % The caption goes in the second argument, and the table contents
 % go in the third argument.
\floatconts
  {tab:glass}%
  {\caption{Significance of results for the \textit{glass} dataset}}%
  {
\begin{tabular}{cl|lll|lll|lll|lll|lll}
             && \multicolumn{3}{c|}{$\epsilon=0.01$} & \multicolumn{3}{c|}{$\epsilon=0.05$} & \multicolumn{3}{c|}{$\epsilon=0.1$} & \multicolumn{3}{c|}{$\epsilon=0.15$} & \multicolumn{3}{c}{$\epsilon=0.2$} \\
\hline
\hline
\hline
\multirow{3}{*}{\rotatebox[origin=c]{90}{$oneC$}}&ip           &            &            &            &            & $-$*       & $-$*       &            & $-$*       & $-$*       &            & $-$        & $-$*       &            & $-$        & $-$*        \\
&ip\_m        &            &            &            & $+$*       &            & $-$*       & $+$*       &            & $-$        & $+$        &            & $-$        & $+$        &            & $-$         \\
&m            &            &            &            & $+$*       & $+$*       &            & $+$*       & $+$        &            & $+$*       & $+$        &            & $+$*       & $+$        &             \\

\hline
\multicolumn{2}{l|}{\textbf{SVM}} & ip         & ip\_m      & m          & ip         & ip\_m      & m          & ip         & ip\_m      & m          & ip         & ip\_m      & m          & ip         & ip\_m      & m           \\
\hline
\multirow{3}{*}{\rotatebox[origin=c]{90}{$avgC$}}&ip           &            &            &            &            &            & $+$*       &            &            & $+$        &            &            &            &            &            &             \\
&ip\_m        &            &            &            &            &            & $+$*       &            &            & $+$*       &            &            & $+$        &            &            &             \\
&m            &            &            &            & $-$*       & $-$*       &            & $-$        & $-$*       &            &            & $-$        &            &            &            &             \\

\hline
\hline
\hline
\multirow{3}{*}{\rotatebox[origin=c]{90}{$oneC$}}&ip           &            &            &            &            &            &            &            &            & $-$        &            & $-$        & $-$        &            & $-$        & $-$         \\
&ip\_m        &            &            &            &            &            &            &            &            & $-$        & $+$        &            & $-$        & $+$        &            & $-$         \\
&m            &            &            &            &            &            &            & $+$        & $+$        &            & $+$        & $+$        &            & $+$        & $+$        &             \\

\hline
\multicolumn{2}{l|}{\textbf{DT}} & ip         & ip\_m      & m          & ip         & ip\_m      & m          & ip         & ip\_m      & m          & ip         & ip\_m      & m          & ip         & ip\_m      & m           \\
\hline
\multirow{3}{*}{\rotatebox[origin=c]{90}{$avgC$}}&ip           &            &            &            &            &            &            &            &            & $-$        &            &            & $-$        &            &            & $-$         \\
&ip\_m        &            &            &            &            &            &            &            &            & $-$        &            &            & $-$        &            &            & $-$         \\
&m            &            &            &            &            &            &            & $+$        & $+$        &            & $+$        & $+$        &            & $+$        & $+$        &             \\

\hline
\hline
\hline
\multirow{3}{*}{\rotatebox[origin=c]{90}{$oneC$}}&ip           &            &            &            &            & $-$        & $-$*       &            & $-$        & $-$*       &            & $-$        & $-$*       &            & $-$        & $-$*        \\
&ip\_m        &            &            &            & $+$        &            & $-$        & $+$        &            & $-$        & $+$        &            & $-$*       & $+$        &            & $-$         \\
&m            &            &            &            & $+$*       & $+$        &            & $+$*       & $+$        &            & $+$*       & $+$*       &            & $+$*       & $+$        &             \\

\hline
\multicolumn{2}{l|}{\textbf{KNN}} & ip         & ip\_m      & m          & ip         & ip\_m      & m          & ip         & ip\_m      & m          & ip         & ip\_m      & m          & ip         & ip\_m      & m           \\
\hline
\multirow{3}{*}{\rotatebox[origin=c]{90}{$avgC$}}&ip           &            &            &            &            & $-$        & $-$        &            & $-$        & $-$        &            & $-$        & $-$        &            &            & $-$         \\
&ip\_m        &            &            &            & $+$        &            & $-$        & $+$        &            & $-$        & $+$        &            & $-$        &            &            &             \\
&m            &            &            &            & $+$        & $+$        &            & $+$        & $+$        &            & $+$        & $+$        &            & $+$        &            &             \\

\hline
\hline
\hline
\multirow{3}{*}{\rotatebox[origin=c]{90}{$oneC$}}&ip           &            &            &            &            &            &            &            &            & $-$*       &            &            & $-$*       &            & $-$*       & $-$*        \\
&ip\_m        &            &            &            &            &            &            &            &            & $-$        &            &            & $-$*       & $+$*       &            & $-$*        \\
&m            &            &            &            &            &            &            & $+$*       & $+$        &            & $+$*       & $+$*       &            & $+$*       & $+$*       &             \\

\hline
\multicolumn{2}{l|}{\textbf{Ada}} & ip         & ip\_m      & m          & ip         & ip\_m      & m          & ip         & ip\_m      & m          & ip         & ip\_m      & m          & ip         & ip\_m      & m           \\
\hline
\multirow{3}{*}{\rotatebox[origin=c]{90}{$avgC$}}&ip           &            &            &            &            &            & $+$*       &            &            & $+$*       &            &            & $+$*       &            &            & $+$*        \\
&ip\_m        &            &            &            &            &            & $+$*       &            &            & $+$*       &            &            & $+$*       &            &            & $+$*        \\
&m            &            &            &            & $-$*       & $-$*       &            & $-$*       & $-$*       &            & $-$*       & $-$*       &            & $-$*       & $-$*       &             \\

\hline
\hline
\hline
\multirow{3}{*}{\rotatebox[origin=c]{90}{$oneC$}}&ip           &            &            &            &            &            & $-$        &            & $-$        & $-$*       &            &            & $-$*       &            & $-$        & $-$*        \\
&ip\_m        &            &            &            &            &            &            & $+$        &            & $-$*       &            &            & $-$*       & $+$        &            & $-$*        \\
&m            &            &            &            & $+$        &            &            & $+$*       & $+$*       &            & $+$*       & $+$*       &            & $+$*       & $+$*       &             \\

\hline
\multicolumn{2}{l|}{\textbf{GNB}} & ip         & ip\_m      & m          & ip         & ip\_m      & m          & ip         & ip\_m      & m          & ip         & ip\_m      & m          & ip         & ip\_m      & m           \\
\hline
\multirow{3}{*}{\rotatebox[origin=c]{90}{$avgC$}}&ip           &            &            &            &            &            & $+$        &            &            & $+$*       &            &            & $+$*       &            &            & $+$*        \\
&ip\_m        &            &            &            &            &            & $+$        &            &            & $+$*       &            &            & $+$*       &            &            & $+$*        \\
&m            &            &            &            & $-$        & $-$        &            & $-$*       & $-$*       &            & $-$*       & $-$*       &            & $-$*       & $-$*       &             \\

\hline
\hline
\hline
\multirow{3}{*}{\rotatebox[origin=c]{90}{$oneC$}}&ip           &            &            &            &            & $-$        &            &            & $-$        & $-$        &            & $-$        &            &            & $-$        &             \\
&ip\_m        &            &            &            & $+$        &            & $+$        & $+$        &            & $+$        & $+$        &            & $+$        & $+$        &            & $+$         \\
&m            &            &            &            &            & $-$        &            & $+$        & $-$        &            &            & $-$        &            &            & $-$        &             \\

\hline
\multicolumn{2}{l|}{\textbf{MPR}} & ip         & ip\_m      & m          & ip         & ip\_m      & m          & ip         & ip\_m      & m          & ip         & ip\_m      & m          & ip         & ip\_m      & m           \\
\hline
\multirow{3}{*}{\rotatebox[origin=c]{90}{$avgC$}}&ip           &            &            &            &            &            & $+$*       &            & $-$        & $+$*       &            & $-$        & $+$*       &            & $-$        & $+$         \\
&ip\_m        &            &            &            &            &            & $+$*       & $+$        &            & $+$*       & $+$        &            & $+$*       & $+$        &            & $+$*        \\
&m            &            &            &            & $-$*       & $-$*       &            & $-$*       & $-$*       &            & $-$*       & $-$*       &            & $-$        & $-$*       &             \\

\hline
\hline
\hline
\multirow{3}{*}{\rotatebox[origin=c]{90}{$oneC$}}&ip           &            &            &            &            & $-$        & $-$*       &            & $-$        & $-$*       &            & $-$        & $-$*       &            & $-$        & $-$*        \\
&ip\_m        &            &            &            & $+$        &            &            & $+$        &            & $-$        & $+$        &            & $-$        & $+$        &            & $-$         \\
&m            &            &            &            & $+$*       &            &            & $+$*       & $+$        &            & $+$*       & $+$        &            & $+$*       & $+$        &             \\

\hline
\multicolumn{2}{l|}{\textbf{RF}} & ip         & ip\_m      & m          & ip         & ip\_m      & m          & ip         & ip\_m      & m          & ip         & ip\_m      & m          & ip         & ip\_m      & m           \\
\hline
\multirow{3}{*}{\rotatebox[origin=c]{90}{$avgC$}}&ip           &            &            &            &            & $-$        & $+$        &            &            & $+$*       &            &            & $+$        &            &            &             \\
&ip\_m        &            &            &            & $+$        &            & $+$*       &            &            & $+$*       &            &            & $+$        &            &            &             \\
&m            &            &            &            & $-$        & $-$*       &            & $-$*       & $-$*       &            & $-$        & $-$        &            &            &            &             \\

\hline
\hline
\hline
\multirow{3}{*}{\rotatebox[origin=c]{90}{$oneC$}}&ip           &            &            &            &            &            & $-$*       &            &            & $-$*       &            &            & $-$        &            &            & $-$         \\
&ip\_m        &            &            &            &            &            & $-$*       &            &            & $-$        &            &            & $-$        &            &            &             \\
&m            &            &            &            & $+$*       & $+$*       &            & $+$*       & $+$        &            & $+$        & $+$        &            & $+$        &            &             \\

\hline
\multicolumn{2}{l|}{\textbf{QDA}} & ip         & ip\_m      & m          & ip         & ip\_m      & m          & ip         & ip\_m      & m          & ip         & ip\_m      & m          & ip         & ip\_m      & m           \\
\hline
\multirow{3}{*}{\rotatebox[origin=c]{90}{$avgC$}}&ip           &            &            &            &            &            &            &            &            & $-$        &            &            &            &            &            &             \\
&ip\_m        &            &            &            &            &            &            &            &            &            &            &            &            &            &            & $+$         \\
&m            &            &            &            &            &            &            & $+$        &            &            &            &            &            &            & $-$        &             \\

\hline
\hline
\hline
\end{tabular}

  }
\end{table}

%\begin{table}[htbp]
\scriptsize
 % The first argument is the label.
 % The caption goes in the second argument, and the table contents
 % go in the third argument.
\floatconts
  {tab:glass}%
  {\caption{glass dataset}}%
  {
\begin{tabular}{cl|lll|lll|lll|lll|lll}
             && \multicolumn{3}{c|}{$\epsilon=0.01$} & \multicolumn{3}{c|}{$\epsilon=0.05$} & \multicolumn{3}{c|}{$\epsilon=0.1$} & \multicolumn{3}{c|}{$\epsilon=0.15$} & \multicolumn{3}{c}{$\epsilon=0.2$} \\
\hline
\hline
\hline
\multirow{3}{*}{\rotatebox[origin=c]{90}{$oneC$}}&ip           &            &            &            &            & $-$*       & $-$*       &            & $-$*       & $-$*       &            & $-$        & $-$*       &            & $-$        & $-$*        \\
&ip\_m        &            &            &            & $+$*       &            & $-$*       & $+$*       &            & $-$        & $+$        &            & $-$        & $+$        &            & $-$         \\
&m            &            &            &            & $+$*       & $+$*       &            & $+$*       & $+$        &            & $+$*       & $+$        &            & $+$*       & $+$        &             \\
\hline
\multicolumn{2}{l|}{\textbf{SVM}} & ip         & ip\_m      & m          & ip         & ip\_m      & m          & ip         & ip\_m      & m          & ip         & ip\_m      & m          & ip         & ip\_m      & m           \\
\hline
\multirow{3}{*}{\rotatebox[origin=c]{90}{$avgC$}}&ip           &            &            &            &            & $-$        & $+$*       &            & $-$        & $+$        &            & $-$        & $+$        &            & $-$        & $+$         \\
&ip\_m        &            &            &            & $+$        &            & $+$*       & $+$        &            & $+$*       & $+$        &            & $+$        & $+$        &            & $+$         \\
&m            &            &            &            & $-$*       & $-$*       &            & $-$        & $-$*       &            & $-$        & $-$        &            & $-$        & $-$        &             \\
\hline
\hline
\hline
\multirow{3}{*}{\rotatebox[origin=c]{90}{$oneC$}}&ip           &            &            &            &            &            & $-$        &            & $-$        & $-$        &            & $-$        & $-$        &            & $-$        & $-$         \\
&ip\_m        &            &            &            &            &            & $-$        & $+$        &            & $-$        & $+$        &            & $-$        & $+$        &            & $-$         \\
&m            &            &            &            & $+$        & $+$        &            & $+$        & $+$        &            & $+$        & $+$        &            & $+$        & $+$        &             \\
\hline
\multicolumn{2}{l|}{\textbf{DT}}  & ip         & ip\_m      & m          & ip         & ip\_m      & m          & ip         & ip\_m      & m          & ip         & ip\_m      & m          & ip         & ip\_m      & m           \\
\hline
\multirow{3}{*}{\rotatebox[origin=c]{90}{$avgC$}}&ip           &            &            &            &            &            & $-$        &            & $-$        & $-$        &            & $-$        & $-$        &            & $-$        & $-$         \\
&ip\_m        &            &            &            &            &            & $-$        & $+$        &            & $-$        & $+$        &            & $-$        & $+$        &            & $-$         \\
&m            &            &            &            & $+$        & $+$        &            & $+$        & $+$        &            & $+$        & $+$        &            & $+$        & $+$        &             \\
\hline
\hline
\hline
\multirow{3}{*}{\rotatebox[origin=c]{90}{$oneC$}}&ip           &            &            &            &            & $-$        & $-$*       &            & $-$        & $-$*       &            & $-$        & $-$*       &            & $-$        & $-$*        \\
&ip\_m        &            &            &            & $+$        &            & $-$        & $+$        &            & $-$        & $+$        &            & $-$*       & $+$        &            & $-$         \\
&m            &            &            &            & $+$*       & $+$        &            & $+$*       & $+$        &            & $+$*       & $+$*       &            & $+$*       & $+$        &             \\
\hline
\multicolumn{2}{l|}{\textbf{KNN}} & ip         & ip\_m      & m          & ip         & ip\_m      & m          & ip         & ip\_m      & m          & ip         & ip\_m      & m          & ip         & ip\_m      & m           \\
\hline
\multirow{3}{*}{\rotatebox[origin=c]{90}{$avgC$}}&ip           &            &            &            &            & $-$        & $-$        &            & $-$        & $-$        &            & $-$        & $-$        &            & $-$        & $-$         \\
&ip\_m        &            &            &            & $+$        &            & $-$        & $+$        &            & $-$        & $+$        &            & $-$        & $+$        &            & $-$         \\
&m            &            &            &            & $+$        & $+$        &            & $+$        & $+$        &            & $+$        & $+$        &            & $+$        & $+$        &             \\
\hline
\hline
\hline
\multirow{3}{*}{\rotatebox[origin=c]{90}{$oneC$}}&ip           &            &            &            &            & $-$        & $-$        &            & $-$        & $-$*       &            & $-$        & $-$*       &            & $-$*       & $-$*        \\
&ip\_m        &            &            &            & $+$        &            & $-$        & $+$        &            & $-$        & $+$        &            & $-$*       & $+$*       &            & $-$*        \\
&m            &            &            &            & $+$        & $+$        &            & $+$*       & $+$        &            & $+$*       & $+$*       &            & $+$*       & $+$*       &             \\
\hline
\multicolumn{2}{l|}{\textbf{Ada}} & ip         & ip\_m      & m          & ip         & ip\_m      & m          & ip         & ip\_m      & m          & ip         & ip\_m      & m          & ip         & ip\_m      & m           \\
\hline
\multirow{3}{*}{\rotatebox[origin=c]{90}{$avgC$}}&ip           &            &            &            &            & $-$        & $+$*       &            & $-$        & $+$*       &            & $-$        & $+$*       &            & $-$        & $+$*        \\
&ip\_m        &            &            &            & $+$        &            & $+$*       & $+$        &            & $+$*       & $+$        &            & $+$*       & $+$        &            & $+$*        \\
&m            &            &            &            & $-$*       & $-$*       &            & $-$*       & $-$*       &            & $-$*       & $-$*       &            & $-$*       & $-$*       &             \\
\hline
\hline
\hline
\multirow{3}{*}{\rotatebox[origin=c]{90}{$oneC$}}&ip           &            &            &            &            & $-$        & $-$        &            & $-$        & $-$*       &            & $-$        & $-$*       &            & $-$        & $-$*        \\
&ip\_m        &            &            &            & $+$        &            & $-$        & $+$        &            & $-$*       & $+$        &            & $-$*       & $+$        &            & $-$*        \\
&m            &            &            &            & $+$        & $+$        &            & $+$*       & $+$*       &            & $+$*       & $+$*       &            & $+$*       & $+$*       &             \\
\hline
\multicolumn{2}{l|}{\textbf{GNB}} & ip         & ip\_m      & m          & ip         & ip\_m      & m          & ip         & ip\_m      & m          & ip         & ip\_m      & m          & ip         & ip\_m      & m           \\
\hline
\multirow{3}{*}{\rotatebox[origin=c]{90}{$avgC$}}&ip           &            &            &            &            & $-$        & $+$        &            & $-$        & $+$*       &            & $-$        & $+$*       &            & $-$        & $+$*        \\
&ip\_m        &            &            &            & $+$        &            & $+$        & $+$        &            & $+$*       & $+$        &            & $+$*       & $+$        &            & $+$*        \\
&m            &            &            &            & $-$        & $-$        &            & $-$*       & $-$*       &            & $-$*       & $-$*       &            & $-$*       & $-$*       &             \\
\hline
\hline
\hline
\multirow{3}{*}{\rotatebox[origin=c]{90}{$oneC$}}&ip           &            &            &            &            & $-$        & $-$        &            & $-$        & $-$        &            & $-$        & $-$        &            & $-$        & $+$         \\
&ip\_m        &            &            &            & $+$        &            & $+$        & $+$        &            & $+$        & $+$        &            & $+$        & $+$        &            & $+$         \\
&m            &            &            &            & $+$        & $-$        &            & $+$        & $-$        &            & $+$        & $-$        &            & $-$        & $-$        &             \\
\hline
\multicolumn{2}{l|}{\textbf{MPR}} & ip         & ip\_m      & m          & ip         & ip\_m      & m          & ip         & ip\_m      & m          & ip         & ip\_m      & m          & ip         & ip\_m      & m           \\
\hline
\multirow{3}{*}{\rotatebox[origin=c]{90}{$avgC$}}&ip           &            &            &            &            & $-$        & $+$*       &            & $-$        & $+$*       &            & $-$        & $+$*       &            & $-$        & $+$         \\
&ip\_m        &            &            &            & $+$        &            & $+$*       & $+$        &            & $+$*       & $+$        &            & $+$*       & $+$        &            & $+$*        \\
&m            &            &            &            & $-$*       & $-$*       &            & $-$*       & $-$*       &            & $-$*       & $-$*       &            & $-$        & $-$*       &             \\
\hline
\hline
\hline
\multirow{3}{*}{\rotatebox[origin=c]{90}{$oneC$}}&ip           &            &            &            &            & $-$        & $-$*       &            & $-$        & $-$*       &            & $-$        & $-$*       &            & $-$        & $-$*        \\
&ip\_m        &            &            &            & $+$        &            & $-$        & $+$        &            & $-$        & $+$        &            & $-$        & $+$        &            & $-$         \\
&m            &            &            &            & $+$*       & $+$        &            & $+$*       & $+$        &            & $+$*       & $+$        &            & $+$*       & $+$        &             \\
\hline
\multicolumn{2}{l|}{\textbf{RF}}  & ip         & ip\_m      & m          & ip         & ip\_m      & m          & ip         & ip\_m      & m          & ip         & ip\_m      & m          & ip         & ip\_m      & m           \\
\hline
\multirow{3}{*}{\rotatebox[origin=c]{90}{$avgC$}}&ip           &            &            &            &            & $-$        & $+$        &            & $-$        & $+$*       &            & $-$        & $+$        &            & $-$        & $+$         \\
&ip\_m        &            &            &            & $+$        &            & $+$*       & $+$        &            & $+$*       & $+$        &            & $+$        & $+$        &            & $+$         \\
&m            &            &            &            & $-$        & $-$*       &            & $-$*       & $-$*       &            & $-$        & $-$        &            & $-$        & $-$        &             \\
\hline
\hline
\hline
\multirow{3}{*}{\rotatebox[origin=c]{90}{$oneC$}}&ip           &            &            &            &            & $-$        & $-$*       &            & $-$        & $-$*       &            & $-$        & $-$        &            & $-$        & $-$         \\
&ip\_m        &            &            &            & $+$        &            & $-$*       & $+$        &            & $-$        & $+$        &            & $-$        & $+$        &            & $-$         \\
&m            &            &            &            & $+$*       & $+$*       &            & $+$*       & $+$        &            & $+$        & $+$        &            & $+$        & $+$        &             \\
\hline
\multicolumn{2}{l|}{\textbf{QDA}} & ip         & ip\_m      & m          & ip         & ip\_m      & m          & ip         & ip\_m      & m          & ip         & ip\_m      & m          & ip         & ip\_m      & m           \\
\hline
\multirow{3}{*}{\rotatebox[origin=c]{90}{$avgC$}}&ip           &            &            &            &            & $-$        & $-$        &            & $-$        & $-$        &            & $-$        & $-$        &            & $-$        & $+$         \\
&ip\_m        &            &            &            & $+$        &            & $-$        & $+$        &            & $-$        & $+$        &            & $-$        & $+$        &            & $+$         \\
&m            &            &            &            & $+$        & $+$        &            & $+$        & $+$        &            & $+$        & $+$        &            & $-$        & $-$        &             \\
\hline
\hline
\hline
\end{tabular}

  }
\end{table}