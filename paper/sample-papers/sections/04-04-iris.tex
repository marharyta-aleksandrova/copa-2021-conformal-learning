\subsection{iris}

Results for iris dataset are presented in \cref{fig:iris} and \cref{tab:iris}. 
First of all, we can notice that the dataset is balanced and each of 3 classes contains the
same number of instances, see \cref{fig:iris:dist-class}. All classifiers perform very good on this dataset with the baseline error ranging from 0.025\% for MRP to 0.06\% for DT, see 
\cref{fig:iris:baseline-error}. The difference between the performance of difference 
non-conformity functions is only slightly visible for $oneC$ of QDA classifier 
\cref{fig:iris:oneC-RF-QDA}, and $avgC$ of MPR classifier \cref{fig:iris:avgC-GNB-MPR}.
This is also supported by the results in \cref{tab:iris}\footnote{\textcolor{red}{If iris is presented in the paper, probably we do not need to show the table, there are no significant differences}}. 
We can notice none of the nonconformity function results in statistically significant 
values of $oneC$ and $avgC$. For many cases, the obtained results are also identical. This 
holds for all setups with $\epsilon=0.01$ and for all setups with KNN classifier. The 
values of $oneC$ metric are identical for Ada, GNB and QDA classifiers, although the values
of $avgC$ are sometimes slightly different.

We can also notice that different classifiers produce conformal predictors of different
efficiency. Generally SVM, GNB, MPR and QDA classifiers produce conformal predictors of 
high efficiency with close values of $oneC$ and $avgC$. These classifiers also have lower
values of baseline error except GNB, see \cref{fig:iris:baseline-error}. On the other hand, 3 classifiers with lower accuracy, DT, ADA and RF produce classifiers of lower efficiency with DT being the worst both in accuracy and efficiency.
KNN classifier, although of having larger accuracy than SVM, also has lower efficiency when
used in conformal framework. Also, we can see that when the value of $\epsilon$ becomes larger than the baseline error of the corresponding baseline classifier, the value of $oneC$ of the relative conformal predictor starts decreasing and the value of $avgC$ is usually less than 1. This means that the confomal predictor starts producing prediction intervals with no labels. The only exception in this case is DT classifier for which the
$b\_err \approx 0.06\%$, however the described tendency occurs only for $\epsilon \ge 0.15$. 
This marks a natural boundary for \textbf{credibility} of the conformal classifiers\footnote{
\textcolor{red}{To check this. Usually credibility together with confidence is defined for every instance. Is there general credibility for the whole dataset?}
}.
When $\epsilon \approx b\_err$, we reach max of $oneC$ and $oneC \approx 1 - b\_err$ and $avgC=1$.

\begin{figure}[htbp]
\floatconts
  {fig:balance}
  {\caption{Results for \textit{balance}: \textit{M} - dashed line,
  \textit{IP} - dash and dot line, \textit{IP\_M} - thin line}}
  {%
    \subfigure[Distribution between classes]{
    \label{fig:balance:dist-class}%
    \includeteximage{plots/balance-dist-class}
    }%
    \qquad
    \subfigure[Baseline error]{\label{fig:square}%
      \includeteximage{plots/balance-baseline-error}
      }
    \qquad
    \subfigure[$oneC$: SVM and DT]{\label{fig:square}%
      \includeteximage{plots/balance-oneC-0}
      %\includeteximage{plots/balance-dist-class}
      }
    \qquad
    \subfigure[$avgC$: SVM and DT]{\label{fig:square}%
      \includeteximage{plots/balance-avgC-0}
      }
    \qquad
    \subfigure[$oneC$: KNN and Ada]{\label{fig:square}%
      \includeteximage{plots/balance-oneC-2}
      }
    \qquad
    \subfigure[$avgC$: KNN and Ada]{\label{fig:square}%
      \includeteximage{plots/balance-avgC-2}
      }
    \qquad
    \subfigure[$oneC$: GNB and MPR]{\label{fig:square}%
      \includeteximage{plots/balance-oneC-4}
      }
    \qquad
    \subfigure[$avgC$: GNB and MPR]{\label{fig:square}%
      \includeteximage{plots/balance-avgC-4}
      }
    \qquad
    \subfigure[$oneC$: RF and QDA]{\label{fig:square}%
      \includeteximage{plots/balance-oneC-6}
      }
    \qquad
    \subfigure[$avgC$: RF and QDA]{\label{fig:square}%
      \includeteximage{plots/balance-avgC-6}
      }
  }
\end{figure}

\include{sections/4-iris-data/table}

%\begin{table}[htbp]
\scriptsize
 % The first argument is the label.
 % The caption goes in the second argument, and the table contents
 % go in the third argument.
\floatconts
  {tab:glass}%
  {\caption{glass dataset}}%
  {
\begin{tabular}{cl|lll|lll|lll|lll|lll}
             && \multicolumn{3}{c|}{$\epsilon=0.01$} & \multicolumn{3}{c|}{$\epsilon=0.05$} & \multicolumn{3}{c|}{$\epsilon=0.1$} & \multicolumn{3}{c|}{$\epsilon=0.15$} & \multicolumn{3}{c}{$\epsilon=0.2$} \\
\hline
\hline
\hline

\hline
\multicolumn{2}{l|}{\textbf{SVM}} & ip         & ip\_m      & m          & ip         & ip\_m      & m          & ip         & ip\_m      & m          & ip         & ip\_m      & m          & ip         & ip\_m      & m           \\
\hline

\hline
\hline
\hline

\hline
\multicolumn{2}{l|}{\textbf{DT}}  & ip         & ip\_m      & m          & ip         & ip\_m      & m          & ip         & ip\_m      & m          & ip         & ip\_m      & m          & ip         & ip\_m      & m           \\
\hline

\hline
\hline
\hline

\hline
\multicolumn{2}{l|}{\textbf{KNN}} & ip         & ip\_m      & m          & ip         & ip\_m      & m          & ip         & ip\_m      & m          & ip         & ip\_m      & m          & ip         & ip\_m      & m           \\
\hline

\hline
\hline
\hline

\hline
\multicolumn{2}{l|}{\textbf{Ada}} & ip         & ip\_m      & m          & ip         & ip\_m      & m          & ip         & ip\_m      & m          & ip         & ip\_m      & m          & ip         & ip\_m      & m           \\
\hline

\hline
\hline
\hline

\hline
\multicolumn{2}{l|}{\textbf{GNB}} & ip         & ip\_m      & m          & ip         & ip\_m      & m          & ip         & ip\_m      & m          & ip         & ip\_m      & m          & ip         & ip\_m      & m           \\
\hline

\hline
\hline
\hline

\hline
\multicolumn{2}{l|}{\textbf{MPR}} & ip         & ip\_m      & m          & ip         & ip\_m      & m          & ip         & ip\_m      & m          & ip         & ip\_m      & m          & ip         & ip\_m      & m           \\
\hline

\hline
\hline
\hline

\hline
\multicolumn{2}{l|}{\textbf{RF}}  & ip         & ip\_m      & m          & ip         & ip\_m      & m          & ip         & ip\_m      & m          & ip         & ip\_m      & m          & ip         & ip\_m      & m           \\
\hline

\hline
\hline
\hline

\hline
\multicolumn{2}{l|}{\textbf{QDA}} & ip         & ip\_m      & m          & ip         & ip\_m      & m          & ip         & ip\_m      & m          & ip         & ip\_m      & m          & ip         & ip\_m      & m           \\
\hline

\hline
\hline
\hline
\end{tabular}

  }
\end{table}