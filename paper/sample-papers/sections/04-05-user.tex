\subsection{user}

Results for \textit{user} dataset are presented in \cref{fig:user} and in \cref{tab:user}. This 
dataset is not perfectly balanced. But 3 our of 4 classes have around 100 instances and the last
class has 2 times less, 50 instances, see \cref{fig:user:dist-class}. 5 classifiers (SVM, DT, 
MPR, RF, QDA) also perform good on this dataset with the baseline error of less than 0.1\%. 3 
other classifier, perform worse: GNB with $b\_err \approx 11\%$, KNN with $b\_err \approx 12\%$ 
and Ada with $b\_err > 30\%$, see \cref{fig:user:baseline-error}. We expect conformal 
classifiers based on these algorithms to give different results for different non-conformity 
functions.
And indeed, if we analyze the more detailed results from \cref{tab:user}, we see that for 
conformal classifiers based on DT, MPR and QDA there is never statistical difference between 
different non-conformity functions and sometimes the produced results are identical. As for SVM,
we observe statistically significant difference in $avgC$ for $\epsilon = 0.05$ (inverse 
probability is better). For RF statistically significant difference is observed only for 
$\epsilon=0.1$ when \textit{ip\_m} non-conformity function outperforms \textit{ip} in terms of 
$oneC$. Note, that even though \textit{m} non-conformity function has the best mean value of 
$one$, it does not outperform \textit{ip} in a statistically significant way.

As for the 3 less accurate classifiers (KNN, Ada and GNB), we can see clear difference in 
\cref{fig:user:oneC-KNN-Ada}, \cref{fig:user:avgC-KNN-Ada} and \cref{fig:user:avgC-GNB-MPR}.
The first thing that we can notice, is that \textit{margin} is the best choice of non-conformity
function for KNN-based conformal predictor: it results in both higher values of $oneC$ and 
$avgC$. This is also confirmed by the results presented in \cref{tab:user}. As for the Ada-based
conformal predictor, the situation is opposite. \textit{Inverse probability} seems to be the
best choice of non-conformity function. Additionally, the combination of both non-conformity
functions which we refer to as \textit{ip\_m} allows to further improve $oneC$ and $avgC$ in
terms of the average results, see \cref{tab:user}. Lastly, for the GNP we can observe the 
pattern described before, that is \textit{margin} tend to result in higher value of $oneC$
and \textit{inverse probability} results in lover values of $avgC$. However, statistically 
significant and visually visible difference is observed only for $avgC$ when $\epsilon=0.5$.
For this value of $\epsilon$ \textit{ip\_m} approach also allows to improve the results provided by \textit{inverse probability} in terms of average performance.

The best conformal predictors are SVM, MPR and QDA and also have very similar behaviour in terms
of $oneC$ and $avgC$. These are the 3 best baseline classification models with $b\_err$ being 
around 5\%. The worst conformal predictor is the one based on Ada classifier. The latter also
has the largest value of the baseline error.

\begin{figure}[htbp]
\floatconts
  {fig:user}
  {\caption{Results for \textit{user} dataset.}}
  {%
    \subfigure[Distribution between classes]{
    \label{fig:user:dist-class}%
    \includeteximage{plots/user-dist-class}
    }%
    \qquad
    \subfigure[Baseline error, $b\_err$]{\label{fig:user:baseline-error}%
      \includeteximage{plots/user-baseline-error}
      }
    \qquad
    \subfigure[$oneC$: SVM and DT]{\label{fig:user:oneC-SVM-DT}%
      \includeteximage{plots/user-oneC-0}
      %\includeteximage{plots/user-dist-class}
      }
    \qquad
    \subfigure[$avgC$: SVM and DT]{\label{fig:user:avgC-SVM-DT}%
      \includeteximage{plots/user-avgC-0}
      }
    \qquad
    \subfigure[$oneC$: KNN and Ada]{\label{fig:user:oneC-KNN-Ada}%
      \includeteximage{plots/user-oneC-2}
      }
    \qquad
    \subfigure[$avgC$: KNN and Ada]{\label{fig:user:avgC-KNN-Ada}%
      \includeteximage{plots/user-avgC-2}
      }
    \qquad
    \subfigure[$oneC$: GNB and MPR]{\label{fig:user:oneC-GNB-MPR}%
      \includeteximage{plots/user-oneC-4}
      }
    \qquad
    \subfigure[$avgC$: GNB and MPR]{\label{fig:user:avgC-GNB-MPR}%
      \includeteximage{plots/user-avgC-4}
      }
    \qquad
    \subfigure[$oneC$: RF and QDA]{\label{fig:user:oneC-RF-QDA}%
      \includeteximage{plots/user-oneC-6}
      }
    \qquad
    \subfigure[$avgC$: RF and QDA]{\label{fig:user:avgC-RF-QDA}%
      \includeteximage{plots/user-avgC-6}
      }
  }
\end{figure}

\begin{table}[htbp]
\scriptsize
 % The first argument is the label.
 % The caption goes in the second argument, and the table contents
 % go in the third argument.
\floatconts
  {tab:user}%
  {\caption{user dataset}}%
  {
\begin{tabular}{cl|lll|lll|lll|lll|lll}
             && \multicolumn{3}{c|}{$\epsilon=0.01$} & \multicolumn{3}{c|}{$\epsilon=0.05$} & \multicolumn{3}{c|}{$\epsilon=0.1$} & \multicolumn{3}{c|}{$\epsilon=0.15$} & \multicolumn{3}{c}{$\epsilon=0.2$} \\
\hline
\hline
\hline
\multirow{3}{*}{\rotatebox[origin=c]{90}{$oneC$}}&ip           &            &            &            &            &            &            &            &            &            &            &            &            &            &            &             \\
&ip\_m        &            &            &            &            &            &            &            &            &            &            &            &            &            &            &             \\
&m            &            &            &            &            &            &            &            &            &            &            &            &            &            &            &             \\

\hline
\multicolumn{2}{l|}{\textbf{SVM}} & ip         & ip\_m      & m          & ip         & ip\_m      & m          & ip         & ip\_m      & m          & ip         & ip\_m      & m          & ip         & ip\_m      & m           \\
\hline
\multirow{3}{*}{\rotatebox[origin=c]{90}{$avgC$}}&ip           &            &            &            &            &            & $+$*       &            &            &            &            &            &            &            &            &             \\
&ip\_m        &            &            &            &            &            & $+$*       &            &            &            &            &            &            &            &            &             \\
&m            &            &            &            & $-$*       & $-$*       &            &            &            &            &            &            &            &            &            &             \\

\hline
\hline
\hline
\multirow{3}{*}{\rotatebox[origin=c]{90}{$oneC$}}&ip           &            &            &            &            &            &            &            &            &            &            &            &            &            &            &             \\
&ip\_m        &            &            &            &            &            &            &            &            &            &            &            &            &            &            &             \\
&m            &            &            &            &            &            &            &            &            &            &            &            &            &            &            &             \\

\hline
\multicolumn{2}{l|}{\textbf{DT}} & ip         & ip\_m      & m          & ip         & ip\_m      & m          & ip         & ip\_m      & m          & ip         & ip\_m      & m          & ip         & ip\_m      & m           \\
\hline
\multirow{3}{*}{\rotatebox[origin=c]{90}{$avgC$}}&ip           &            &            &            &            &            &            &            &            &            &            &            &            &            &            &             \\
&ip\_m        &            &            &            &            &            &            &            &            &            &            &            &            &            &            &             \\
&m            &            &            &            &            &            &            &            &            &            &            &            &            &            &            &             \\

\hline
\hline
\hline
\multirow{3}{*}{\rotatebox[origin=c]{90}{$oneC$}}&ip           &            &            &            &            &            & $-$        &            &            & $-$*       &            &            & $-$        &            &            & $-$         \\
&ip\_m        &            &            &            &            &            & $-$        &            &            & $-$*       &            &            & $-$        &            &            & $-$         \\
&m            &            &            &            & $+$        & $+$        &            & $+$*       & $+$*       &            & $+$        & $+$        &            & $+$        & $+$        &             \\

\hline
\multicolumn{2}{l|}{\textbf{KNN}} & ip         & ip\_m      & m          & ip         & ip\_m      & m          & ip         & ip\_m      & m          & ip         & ip\_m      & m          & ip         & ip\_m      & m           \\
\hline
\multirow{3}{*}{\rotatebox[origin=c]{90}{$avgC$}}&ip           &            &            &            &            &            & $-$        &            &            & $-$*       &            &            &            &            &            &             \\
&ip\_m        &            &            &            &            &            & $-$        &            &            & $-$*       &            &            &            &            &            &             \\
&m            &            &            &            & $+$        & $+$        &            & $+$*       & $+$*       &            &            &            &            &            &            &             \\

\hline
\hline
\hline
\multirow{3}{*}{\rotatebox[origin=c]{90}{$oneC$}}&ip           &            &            &            &            &            &            &            &            &            &            &            & $+$        &            &            &             \\
&ip\_m        &            &            &            &            &            &            &            &            &            &            &            & $+$        &            &            &             \\
&m            &            &            &            &            &            &            &            &            &            & $-$        & $-$        &            &            &            &             \\

\hline
\multicolumn{2}{l|}{\textbf{Ada}} & ip         & ip\_m      & m          & ip         & ip\_m      & m          & ip         & ip\_m      & m          & ip         & ip\_m      & m          & ip         & ip\_m      & m           \\
\hline
\multirow{3}{*}{\rotatebox[origin=c]{90}{$avgC$}}&ip           &            &            &            &            &            & $+$*       &            &            & $+$*       &            &            & $+$*       &            &            & $+$         \\
&ip\_m        &            &            &            &            &            & $+$*       &            &            & $+$*       &            &            & $+$*       &            &            & $+$*        \\
&m            &            &            &            & $-$*       & $-$*       &            & $-$*       & $-$*       &            & $-$*       & $-$*       &            & $-$        & $-$*       &             \\

\hline
\hline
\hline
\multirow{3}{*}{\rotatebox[origin=c]{90}{$oneC$}}&ip           &            &            &            &            &            &            &            &            &            &            &            &            &            &            &             \\
&ip\_m        &            &            &            &            &            &            &            &            &            &            &            &            &            &            &             \\
&m            &            &            &            &            &            &            &            &            &            &            &            &            &            &            &             \\

\hline
\multicolumn{2}{l|}{\textbf{GNB}} & ip         & ip\_m      & m          & ip         & ip\_m      & m          & ip         & ip\_m      & m          & ip         & ip\_m      & m          & ip         & ip\_m      & m           \\
\hline
\multirow{3}{*}{\rotatebox[origin=c]{90}{$avgC$}}&ip           &            &            &            &            &            & $+$*       &            &            & $+$        &            &            &            &            &            &             \\
&ip\_m        &            &            &            &            &            & $+$*       &            &            & $+$        &            &            &            &            &            &             \\
&m            &            &            &            & $-$*       & $-$*       &            & $-$        & $-$        &            &            &            &            &            &            &             \\

\hline
\hline
\hline
\multirow{3}{*}{\rotatebox[origin=c]{90}{$oneC$}}&ip           &            &            &            &            &            &            &            &            &            &            &            &            &            &            &             \\
&ip\_m        &            &            &            &            &            &            &            &            &            &            &            &            &            &            &             \\
&m            &            &            &            &            &            &            &            &            &            &            &            &            &            &            &             \\

\hline
\multicolumn{2}{l|}{\textbf{MPR}} & ip         & ip\_m      & m          & ip         & ip\_m      & m          & ip         & ip\_m      & m          & ip         & ip\_m      & m          & ip         & ip\_m      & m           \\
\hline
\multirow{3}{*}{\rotatebox[origin=c]{90}{$avgC$}}&ip           &            &            &            &            &            &            &            &            &            &            &            &            &            &            &             \\
&ip\_m        &            &            &            &            &            &            &            &            &            &            &            &            &            &            &             \\
&m            &            &            &            &            &            &            &            &            &            &            &            &            &            &            &             \\

\hline
\hline
\hline
\multirow{3}{*}{\rotatebox[origin=c]{90}{$oneC$}}&ip           &            &            &            &            & $-$        & $-$        &            & $-$*       &            &            &            &            &            &            &             \\
&ip\_m        &            &            &            & $+$        &            &            & $+$*       &            &            &            &            &            &            &            &             \\
&m            &            &            &            & $+$        &            &            &            &            &            &            &            &            &            &            &             \\

\hline
\multicolumn{2}{l|}{\textbf{RF}} & ip         & ip\_m      & m          & ip         & ip\_m      & m          & ip         & ip\_m      & m          & ip         & ip\_m      & m          & ip         & ip\_m      & m           \\
\hline
\multirow{3}{*}{\rotatebox[origin=c]{90}{$avgC$}}&ip           &            &            &            &            &            & $+$        &            &            &            &            &            &            &            &            &             \\
&ip\_m        &            &            &            &            &            & $+$        &            &            &            &            &            &            &            &            &             \\
&m            &            &            &            & $-$        & $-$        &            &            &            &            &            &            &            &            &            &             \\

\hline
\hline
\hline
\multirow{3}{*}{\rotatebox[origin=c]{90}{$oneC$}}&ip           &            &            &            &            &            &            &            &            &            &            &            &            &            &            &             \\
&ip\_m        &            &            &            &            &            &            &            &            &            &            &            &            &            &            &             \\
&m            &            &            &            &            &            &            &            &            &            &            &            &            &            &            &             \\

\hline
\multicolumn{2}{l|}{\textbf{QDA}} & ip         & ip\_m      & m          & ip         & ip\_m      & m          & ip         & ip\_m      & m          & ip         & ip\_m      & m          & ip         & ip\_m      & m           \\
\hline
\multirow{3}{*}{\rotatebox[origin=c]{90}{$avgC$}}&ip           &            &            &            &            &            & $+$        &            &            &            &            &            &            &            &            &             \\
&ip\_m        &            &            &            &            &            & $+$        &            &            &            &            &            &            &            &            &             \\
&m            &            &            &            & $-$        & $-$        &            &            &            &            &            &            &            &            &            &             \\

\hline
\hline
\hline
\end{tabular}

  }
\end{table}

%\begin{table}[htbp]
\scriptsize
 % The first argument is the label.
 % The caption goes in the second argument, and the table contents
 % go in the third argument.
\floatconts
  {tab:user}%
  {\caption{user dataset}}%
  {
\begin{tabular}{cl|lll|lll|lll|lll|lll}
             && \multicolumn{3}{c|}{$\epsilon=0.01$} & \multicolumn{3}{c|}{$\epsilon=0.05$} & \multicolumn{3}{c|}{$\epsilon=0.1$} & \multicolumn{3}{c|}{$\epsilon=0.15$} & \multicolumn{3}{c}{$\epsilon=0.2$} \\
\hline
\hline
\hline
\multirow{3}{*}{\rotatebox[origin=c]{90}{$oneC$}}&ip           &            &            &            &            &            & $+$        &            &            & $+$        &            &            & $-$        &            &            & $+$         \\
&ip\_m        &            &            &            &            &            & $+$        &            &            & $+$        &            &            & $-$        &            &            & $+$         \\
&m            &            &            &            & $-$        & $-$        &            & $-$        & $-$        &            & $+$        & $+$        &            & $-$        & $-$        &             \\
\hline
\multicolumn{2}{l|}{\textbf{SVM}} & ip         & ip\_m      & m          & ip         & ip\_m      & m          & ip         & ip\_m      & m          & ip         & ip\_m      & m          & ip         & ip\_m      & m           \\
\hline
\multirow{3}{*}{\rotatebox[origin=c]{90}{$avgC$}}&ip           &            &            &            &            &            & $+$*       &            &            & $-$        &            &            & $+$        &            &            & $-$         \\
&ip\_m        &            &            &            &            &            & $+$*       &            &            & $-$        &            &            & $+$        &            &            & $-$         \\
&m            &            &            &            & $-$*       & $-$*       &            & $+$        & $+$        &            & $-$        & $-$        &            & $+$        & $+$        &             \\
\hline
\hline
\hline
\multirow{3}{*}{\rotatebox[origin=c]{90}{$oneC$}}&ip           &            &            &            &            &            & $-$        &            &            & $-$        &            &            & $+$        &            &            & $+$         \\
&ip\_m        &            &            &            &            &            & $-$        &            &            & $-$        &            &            & $+$        &            &            & $+$         \\
&m            &            &            &            & $+$        & $+$        &            & $+$        & $+$        &            & $-$        & $-$        &            & $-$        & $-$        &             \\
\hline
\multicolumn{2}{l|}{\textbf{DT}}  & ip         & ip\_m      & m          & ip         & ip\_m      & m          & ip         & ip\_m      & m          & ip         & ip\_m      & m          & ip         & ip\_m      & m           \\
\hline
\multirow{3}{*}{\rotatebox[origin=c]{90}{$avgC$}}&ip           &            &            &            &            &            & $-$        &            &            & $-$        &            &            & $-$        &            &            & $-$         \\
&ip\_m        &            &            &            &            &            & $-$        &            &            & $-$        &            &            & $-$        &            &            & $-$         \\
&m            &            &            &            & $+$        & $+$        &            & $+$        & $+$        &            & $+$        & $+$        &            & $+$        & $+$        &             \\
\hline
\hline
\hline
\multirow{3}{*}{\rotatebox[origin=c]{90}{$oneC$}}&ip           &            &            &            &            &            & $-$        &            &            & $-$*       &            &            & $-$        &            &            & $-$         \\
&ip\_m        &            &            &            &            &            & $-$        &            &            & $-$*       &            &            & $-$        &            &            & $-$         \\
&m            &            &            &            & $+$        & $+$        &            & $+$*       & $+$*       &            & $+$        & $+$        &            & $+$        & $+$        &             \\
\hline
\multicolumn{2}{l|}{\textbf{KNN}} & ip         & ip\_m      & m          & ip         & ip\_m      & m          & ip         & ip\_m      & m          & ip         & ip\_m      & m          & ip         & ip\_m      & m           \\
\hline
\multirow{3}{*}{\rotatebox[origin=c]{90}{$avgC$}}&ip           &            &            &            &            &            & $-$        &            &            & $-$*       &            &            & $-$        &            &            & $-$         \\
&ip\_m        &            &            &            &            &            & $-$        &            &            & $-$*       &            &            & $-$        &            &            & $-$         \\
&m            &            &            &            & $+$        & $+$        &            & $+$*       & $+$*       &            & $+$        & $+$        &            & $+$        & $+$        &             \\
\hline
\hline
\hline
\multirow{3}{*}{\rotatebox[origin=c]{90}{$oneC$}}&ip           &            &            &            &            &            & $+$        &            & $-$        & $+$        &            & $-$        & $+$        &            & $-$        & $-$         \\
&ip\_m        &            &            &            &            &            & $+$        & $+$        &            & $+$        & $+$        &            & $+$        & $+$        &            & $-$         \\
&m            &            &            &            & $-$        & $-$        &            & $-$        & $-$        &            & $-$        & $-$        &            & $+$        & $+$        &             \\
\hline
\multicolumn{2}{l|}{\textbf{Ada}} & ip         & ip\_m      & m          & ip         & ip\_m      & m          & ip         & ip\_m      & m          & ip         & ip\_m      & m          & ip         & ip\_m      & m           \\
\hline
\multirow{3}{*}{\rotatebox[origin=c]{90}{$avgC$}}&ip           &            &            &            &            &            & $+$*       &            & $-$        & $+$*       &            & $-$        & $+$*       &            & $-$        & $+$         \\
&ip\_m        &            &            &            &            &            & $+$*       & $+$        &            & $+$*       & $+$        &            & $+$*       & $+$        &            & $+$*        \\
&m            &            &            &            & $-$*       & $-$*       &            & $-$*       & $-$*       &            & $-$*       & $-$*       &            & $-$        & $-$*       &             \\
\hline
\hline
\hline
\multirow{3}{*}{\rotatebox[origin=c]{90}{$oneC$}}&ip           &            &            &            &            & $-$        & $-$        &            &            & $-$        &            &            & $-$        &            &            &             \\
&ip\_m        &            &            &            & $+$        &            & $-$        &            &            & $-$        &            &            & $-$        &            &            &             \\
&m            &            &            &            & $+$        & $+$        &            & $+$        & $+$        &            & $+$        & $+$        &            &            &            &             \\
\hline
\multicolumn{2}{l|}{\textbf{GNB}} & ip         & ip\_m      & m          & ip         & ip\_m      & m          & ip         & ip\_m      & m          & ip         & ip\_m      & m          & ip         & ip\_m      & m           \\
\hline
\multirow{3}{*}{\rotatebox[origin=c]{90}{$avgC$}}&ip           &            &            &            &            & $-$        & $+$*       &            &            & $+$        &            &            & $+$        &            &            & $+$         \\
&ip\_m        &            &            &            & $+$        &            & $+$*       &            &            & $+$        &            &            & $+$        &            &            & $+$         \\
&m            &            &            &            & $-$*       & $-$*       &            & $-$        & $-$        &            & $-$        & $-$        &            & $-$        & $-$        &             \\
\hline
\hline
\hline
\multirow{3}{*}{\rotatebox[origin=c]{90}{$oneC$}}&ip           &            &            &            &            &            & $+$        &            &            & $-$        &            &            & $+$        &            &            & $+$         \\
&ip\_m        &            &            &            &            &            & $+$        &            &            & $-$        &            &            & $+$        &            &            & $+$         \\
&m            &            &            &            & $-$        & $-$        &            & $+$        & $+$        &            & $-$        & $-$        &            & $-$        & $-$        &             \\
\hline
\multicolumn{2}{l|}{\textbf{MPR}} & ip         & ip\_m      & m          & ip         & ip\_m      & m          & ip         & ip\_m      & m          & ip         & ip\_m      & m          & ip         & ip\_m      & m           \\
\hline
\multirow{3}{*}{\rotatebox[origin=c]{90}{$avgC$}}&ip           &            &            &            &            &            & $+$        &            &            & $+$        &            &            & $-$        &            &            & $-$         \\
&ip\_m        &            &            &            &            &            & $+$        &            &            & $+$        &            &            & $-$        &            &            & $-$         \\
&m            &            &            &            & $-$        & $-$        &            & $-$        & $-$        &            & $+$        & $+$        &            & $+$        & $+$        &             \\
\hline
\hline
\hline
\multirow{3}{*}{\rotatebox[origin=c]{90}{$oneC$}}&ip           &            &            &            &            & $-$        & $-$        &            & $-$*       & $-$        &            & $-$        & $-$        &            & $-$        & $-$         \\
&ip\_m        &            &            &            & $+$        &            & $+$        & $+$*       &            & $-$        & $+$        &            & $-$        & $+$        &            & $-$         \\
&m            &            &            &            & $+$        & $-$        &            & $+$        & $+$        &            & $+$        & $+$        &            & $+$        & $+$        &             \\
\hline
\multicolumn{2}{l|}{\textbf{RF}}  & ip         & ip\_m      & m          & ip         & ip\_m      & m          & ip         & ip\_m      & m          & ip         & ip\_m      & m          & ip         & ip\_m      & m           \\
\hline
\multirow{3}{*}{\rotatebox[origin=c]{90}{$avgC$}}&ip           &            &            &            &            & $-$        & $+$        &            & $-$        & $-$        &            & $-$        & $+$        &            & $-$        & $+$         \\
&ip\_m        &            &            &            & $+$        &            & $+$        & $+$        &            & $+$        & $+$        &            & $+$        & $+$        &            & $+$         \\
&m            &            &            &            & $-$        & $-$        &            & $+$        & $-$        &            & $-$        & $-$        &            & $-$        & $-$        &             \\
\hline
\hline
\hline
\multirow{3}{*}{\rotatebox[origin=c]{90}{$oneC$}}&ip           &            &            &            &            &            &            &            &            & $-$        &            &            & $-$        &            &            &             \\
&ip\_m        &            &            &            &            &            &            &            &            & $-$        &            &            & $-$        &            &            &             \\
&m            &            &            &            &            &            &            & $+$        & $+$        &            & $+$        & $+$        &            &            &            &             \\
\hline
\multicolumn{2}{l|}{\textbf{QDA}} & ip         & ip\_m      & m          & ip         & ip\_m      & m          & ip         & ip\_m      & m          & ip         & ip\_m      & m          & ip         & ip\_m      & m           \\
\hline
\multirow{3}{*}{\rotatebox[origin=c]{90}{$avgC$}}&ip           &            &            &            &            &            & $+$        &            &            & $+$        &            &            & $+$        &            &            &             \\
&ip\_m        &            &            &            &            &            & $+$        &            &            & $+$        &            &            & $+$        &            &            &             \\
&m            &            &            &            & $-$        & $-$        &            & $-$        & $-$        &            & $-$        & $-$        &            &            &            &             \\
\hline
\hline
\hline
\end{tabular}

  }
\end{table}


