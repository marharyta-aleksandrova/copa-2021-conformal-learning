\section{Experimental setup}
\label{sec:setup}

%\textcolor{red}{Give names (formulas) to all rows in \cref{tab:datasets-and-results} to reference them in the text.}

To perform experimental analysis, we used the implementation of conformal
predictors available from
\textit{nonconformist}\footnote{\url{https://github.com/donlnz/nonconformist}} Python library. 
We followed the general experimental setup from the original paper by \citet{johansson2017model}.
That is we used 10x10-fold cross-validation with 90\% of the data used for 
training and validation of the model, and 10\% used for testing.
The training dataset was further split into a proper training set and a 
calibration set in proportion 4:1, i.e., 80\% of the training set was used for
actual training of the classification model, and the rest 20\% were used 
for calibration.
All the results reported below are averaged over the 10x10 folds.

In the original study, the authors used
21 publicly available multi-class datasets from UCI repository~\cite{Dua:2019}.
In this paper, we present not aggregated, but detailed results for every analyzed
dataset. That is why we chose 9 representative datasets with different
characteristics from the original list of 21 ones. 
The general information about these datasets, such as the number of instances, attributes, and defined classes is given in the first section of \cref{tab:datasets-and-results}.

\begin{sidewaystable}[htbp]
%\scriptsize
\floatconts
  {tab:datasets-and-results}
  {\caption{Characteristics of the datasets and summarization of results}}
  {
\begin{tabular}{l|l|l|l|l|l|l|l|l|l|l|l}
                             \multicolumn{3}{r|}{\textbf{Datasets} }                                & \textbf{iris} & \textit{user} & \textbf{glass} & \textit{cars} & \textbf{wave} & \textbf{balance} & \textit{wineR} & \textit{wineW} & \textit{yeast}            \\
\hline
\hline
\hline
\multirow{4}{*}{1} & \multirow{4}{*}{\rotatebox[origin=c]{90}{info}} & \# instances                    & 150           & 403           & 214            & 1728          & 5000          & 625              & 1599           & 4898           & 1484                      \\
                            & & \# attributes                   & 4             & 5             & 9              & 6             & 21            & 4                & 11             & 11             & 8                         \\
                            & & \# classes                      & 3             & 4             & 6              & 4             & 3             & 3                & 6              & 7              & 10                        \\
                            & & balanced                        & Y             & mostly        & N              & N             & Y             & N                & N              & N              & N                         \\
\hline
\hline
\hline
\multirow{2}{*}{2} & \multicolumn{2}{l|}{$b\_err$ range, \%}                           & 2-6           & 5-32          & 24-92          & 7-96          & 13-25         & 3-21             & 38-50          & 45-58          & 40-87                     \\
& \multicolumn{2}{l|}{$b\_err$ median, \%}                          & 4             & 7             & 48             & 13            & 17            & 10               & 45             & 53             & 44                        \\
\hline
\hline
\hline
\multirow{3}{*}{3} & \multirow{3}{*}{\rotatebox[origin=c]{90}{ $E\_oneC$}} & mean                      & 0.98          & 0.93          & 0.63           & 0.88          & 0.90          & 0.96             & 0.52           & 0.50           & 0.56                      \\
                            & & mean-std                        & 0.00          & 0.00          & 0.02           & 0.00          & 0.00          & 0.00             & 0.03           & 0.02           & 0.02                      \\
                            & & corr. $b\_acc$               & 0.92          & 0.98          & 0.69           & 0.93          & 0.92          & 0.80             & 0.27           & 0.78           & 0.97                      \\
\hline
\hline
\hline
\multirow{4}{*}{4}& \multirow{2}{*}{\rotatebox[origin=c]{90}{thres.}}      & $oneC$,\%                       & 5             & 15            & 75             & 47.5          & 25            & 67.5             & 72.5           & 65             & 77.5                      \\
                            & & $avgC$,\%                       & 2.5           & 22.5          & 67.5           & 37.5          & 27.5          & 42.5             & 80             & 77.5           & 82.5                      \\\cline{2-12}
%\hline
&\multirow{2}{*}{\rotatebox[origin=c]{90}{stat.}}       & $oneC$,\%                       &               & 5             & 50             & 45            & 30            & 55               & 82.5           & 85             & 85                        \\
                            & & $avgC$,\%                       &               & 17.5          & 37.5           & 45            & 47.5          & 50               & 82.5           & 77.5           & 77.5                      \\
\hline
\hline
\hline
\multirow{6}{*}{5} &\multirow{6}{*}{\rotatebox[origin=c]{90}{patterns}}    & \multirow{3}{*}{\textit{M} is the best} &               & {\scriptsize \textbf{KNN}}           & {\scriptsize \textbf{\textbf{KNN}},}        & {\scriptsize \textbf{\textbf{KNN}}}           & {\scriptsize Ada}         & {\scriptsize \textbf{\textbf{KNN}}, DT, RF,}    & {\scriptsize \textbf{\textbf{KNN}}}            & { \scriptsize \textbf{\textbf{KNN}}}            & {\scriptsize \textbf{\textbf{KNN}}(0.05, 0.1, 0.15),}     \\
                             &&                                 &               &               & {\scriptsize DT}               &               & {\scriptsize (0.1, 
                             0.15,}      & {\scriptsize GNP, MPR,}        & \scriptsize{(0.05, 0.1),}    & {\scriptsize (0.05, 0.1)}     & {\scriptsize DT(0.05), Ada(0.15,} \\
                             &&                                 &               &               &                &               & {\scriptsize 0.20)}   & {\scriptsize Ada, QDA }         & {\scriptsize DT(0.05) }      &                & {\scriptsize  0.2), QDA(0.15, 0.2)}            \\\cline{3-12}
                             && \textit{IP} is the best                      &               & {\scriptsize Ada(0.15)}     &                &               & {\scriptsize Ada(0.01)}     &                  &                &                &                           \\\cline{3-12}
                             && \textit{IP\_M} is the best                   &               &               & {\scriptsize MPR}            & {\scriptsize RF}           & {\scriptsize RF }          &                  &                &                &                           \\\cline{3-12}
                             && \textit{IP\_M} $>$ \textit{IP}               & Y             & Y             & Y              & Y             & Y             & Y                & Y              & Y              & Y\\    
\hline
\hline
\hline
%6 & \multicolumn{2}{l|}{credibility \& $b\_err$}                       & Y              & Y          %   &                & Y             & Y             & Y                &                &            %    &                           \\
%\hline
%\hline
%\hline
%                             7 & \multicolumn{2}{c|}{$b\_acc \uparrow \Rightarrow $ eff. $\uparrow$}                & ?             & ?             & ?              & ?             & ?             & ?                & ?              & ?              & ?\\
\end{tabular}
  }
\end{sidewaystable}


The original study by \citet{johansson2017model} analyzed the performance of 
conformal classifiers based on the ANN classification model.
In this paper, we aim to further extend this analysis and use 8 different
classification algorithms as baseline models: Support Vector Machine (SVM), Decision Tree (DT), $k$-Nearest Neighbours (KNN), AdaBoost (Ada), Gaussian Naive Bayes (GNB), Multilayer Perceptron (MPR), Random Forest (RF) and Quadratic Discriminant Analysis (QDA). We used implementations of these algorithms
available from the  \textit{scikit-learn} Python library. In \cref{tab:algo-params},
we summarize the input parameters of these algorithms unless the default values 
are used.

\begin{table}[htbp]
 % The first argument is the label.
 % The caption goes in the second argument, and the table contents
 % go in the third argument.
\floatconts
  {tab:algo-params}%
  {\caption{Input parameters of classification algorithms}}%
  {
\begin{tabular}{l|l}
Algorithm & Input parameters                                               \\
\hline
SVM       & probability=True                                               \\
DT        & min\_samples\_split=max(5,  5\% of proper training dataset) \\
KNN       & n\_neighbors=5                                                 \\
MPR       & alpha=1, max\_iter=1000                                        \\
RF        & n\_estimators=10, min\_samples\_split=0                         
\end{tabular}
  }
\end{table}

Different classifiers perform differently on different datasets.
We demonstrate this in the second section of \cref{tab:datasets-and-results}.
To calculate the corresponding values, we used the same 10x10-fold cross-validation but without splitting the training set into a proper training set and a
validation set.
\textit{$b\_err$ range} from the first row of this section demonstrates the range of
errors produced by all 8 classification algorithms in the baseline mode\footnote{
In this text we use the \textit{baseline mode} to refer to the standard (non-conformal) prediction.
}. We can 
notice that some datasets are easier to classify, for example, the \verb|iris| dataset for which  the maximum error is 6\%. 
At the same time, other datasets are more difficult, for example, \verb|wineW| for which none of the classifiers can produce error less than 45\%.
The performance of classifiers is not uniformly distributed within the given ranges.
This can be seen from the median of baseline error distribution, see row \textit{$b\_err$ median}.
For example, for the  \verb|cars| dataset different classifiers result in errors ranging from 7\% to 96\%. However, the median value of 13 shows that half of them perform relatively well.

%For our analysis we used 3 nonconformity functions: 1) inverse probability, \textit{IP} (also known as \textit{hinge}) defined by the following formula $\Delta \left[h \left( \vec{x_i}, y_i \right) \right] = 1 - \hat{P}_h\left(y_i | \vec{x_i} \right)$; 2) margin (\textit{M}) defined by equation $\Delta \left[h \left( \vec{x_i}, y_i \right) \right] = \max_{y \neq y_i} \hat{P}_h\left(y | \vec{x_i} \right) - \hat{P}_h\left(y_i | \vec{x_i} \right)$; 3) and the combination of \textit{inverse probability} and \textit{margin} as defined in \cref{sec:our-approach}.

All experimental evaluations were performed for 5 different values of significance
level $\epsilon \in \left\{ 0.01, 0.05, 0.1, 0.15, 0.20 \right\}$. For every 
combination of dataset, baseline classification algorithm and $\epsilon$, we 
calculated the values of $oneC$ and $avgC$ with 2 different nonconformity 
functions (\textit{IP} and \textit{M}) and their combination \textit{IP\_M}.
After that, the results were compared to see if any of the nonconformity 
functions or their combination 
results in a more efficient conformal predictor. 
Due to the space limitations of this paper, we present 
detailed results only for 4 datasets highlighted in bold in \cref{tab:datasets-and-results}. 
However, experimental results for all datasets together with the 
code used for the experimentation
are available in the related git repository\footnote{
\url{https://github.com/marharyta-aleksandrova/copa-2021-conformal-learning}
}.

%\citet{johansson2017model}

%\citep{johansson2017model}